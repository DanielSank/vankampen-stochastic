\leveldown{Normalization of conditional probability - pg. 11}

\leveldown{Problem}

Prove and interpret the normalization of the conditional probability

\begin{equation*}
  \int P_{A|B}(a, b) da = 1 \, .
\end{equation*}

\levelstay{Solution}

Use Bayes's rule:
\begin{align*}
  \int P_{A|B}(a, b) da
  &= \frac{1}{P_B(b)} \int P_{A,B}(a, b) \, da \\
  &= \frac{P_B(b)}{P_B(b)} \\
  &= 1 \, .
\end{align*}
The normalization of the conditional probability $P_{A|B}$ can be interpreted as the following English sentences: ``For any value of the variables $B$, the sum of the probabilities for all values of the variables $A$ is one.
In other words, after fixing $B$, the remaining variables are still a probability distribution.''
