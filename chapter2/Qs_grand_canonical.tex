\leveldown{$Q_s$ functions in the grand-canonical ensemble - pg. 32}

\leveldown{Problem}

Write explicitly the functions $Q_s$ for the grand canonical ensemble of an ideal gas in a fixed volume.

\levelstay{Solution}

In the context of the grand canonical ensemble with only one chemical species the functions $Q_N$ are probability distributions over the state of subsystems of the ensemble of volume V. Such subsystems are characterized by number of particles N and total energy $U_m$. Thus $Q_N(U_m)$ is the probability that a subsystem has N particles and is in energy state $U_m$. In the grand-canonical ensemble :
\begin{equation}
U_m =TS-PV+N\mu 
\end{equation}
The probability of a state is determined by its entropy (because entropy measures the number of possible configurations out of the total number of configurations available). In this case
\begin{equation}
Q_N(U_m)=e^{-S/k}
\end{equation}
where S is the entropy consistent with the number of particles and the energy of the system.
\begin{equation}
Q_N(U_m)=e^{{\displaystyle\frac{PV}{kT}}-{\displaystyle\frac{U_m}{kT}}+{\displaystyle\frac{N \mu}{kT}}}
\end{equation}