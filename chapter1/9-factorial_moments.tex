\leveldown{Factorial moments - pg. 9}

\leveldown{Problem}

When $X$ only takes the values $0, 1, 2,\ldots$ one defines the \emph{factorial moments} $\phi_m$ by $\phi_0 = 1$ and
\begin{equation*}
  \phi_m = \left \langle X (X-1) (X-2) \cdots (X-m+1) \right\rangle \quad (m \geq 1) \, .
\end{equation*}
Show that they are also generated by $F$, viz.,
\begin{equation*}
  F(1 - x) = \sum_{m=1}^\infty \frac{(-x)^m}{m!} \theta_m \, .
\end{equation*}

\levelstay{Solution}

Let's recall the meaning of generating function.
The moment generating function $G$, defined by the equation $G(k) = \avg{e^{ikx}}$, has the property that
\begin{equation*}
  (D^nG)(k=0) = i^n \avg{x^n} \equiv i^n \mu_n
  \, .
\end{equation*}
As the $n^\text{th}$ derivative at $k=0$ is $i^n \mu_n$, we can Taylor expand $G$ about $k=0$ as
\begin{equation*}
  G(k) = \sum_{n=0}^\infty \frac{k^n}{n!} i^n \mu_n
  \, .
\end{equation*}


Define a function $F$ as
\begin{equation*}
  F(k) = \left \langle k^x \right \rangle
  \, .
\end{equation*}
Differentiating $F$ and evaluating at $k=1$ gives us the factorial moments
\begin{align*}
  (D^m F)(k)
  &= \left\langle x(x-1)(x-2)\cdots(x-m+1) k^{x-m}\right\rangle \\
  (D^m F)(1)
  &= \left\langle x(x-1)(x-2)\cdots(x-m+1)\right\rangle \\
  &= \phi_m \, .
\end{align*}
  As the $n^\text{th}$ derivative of $F$ at $k=1$ is $\phi_n$, we can Taylor expand $F$ as
\begin{equation*}
  F(k) = \sum_{n=0}^\infty \frac{(k-1)^n}{n!} \phi_n
\end{equation*}
which can also be written as
\begin{equation*}
  F(1-k) = \sum_{n=0}^\infty \frac{(-k)^n}{n!} \phi_n
  \, .
\end{equation*}
I have no idea why the book writes the latter form.
