\leveldown{Mean square of N in given interval - pg. 32}

\leveldown{Problem}

Show that the mean square of N (number of dots) is given by the expression (1.7) in the book.

\levelstay{Solution}

The number $N$ of dots occurring in time follows the sequence
\begin{equation}
  N \rightarrow \left\{ 0, \chi(\tau_1), \chi(\tau_1) + \chi(\tau_2), \chi(\tau_1) + \chi(\tau_1) + \chi(\tau_3),\dots \right\}
  \, .
\end{equation}
Therefore
\begin{equation}
  \avgang{N^2}
  = \avgang{\left[ \sum_{\sigma=1}^s \chi(\tau_{\sigma})\right]^2}
  = \sum_{s=1}^{\infty} \frac{1}{s!} \int_{-\infty}^{\infty} \left[\sum_{\sigma=1}^s \chi(\tau_{\sigma})\right]^2Q_s(\tau_1,\tau_2,...,\tau_s)d\tau_1d\tau_2 \ldots \, d\tau_s
  \, .
\end{equation}
The squared sum can be split up as
\begin{align*}
  \left[ \sum_{\sigma=1}^s \chi(\tau_{\sigma}) \right]^2
  &=\sum_{\sigma=1}^1 \chi^2(\tau_{\sigma})+2\chi(\tau_1)\chi(\tau_2) + \ldots + 2 \chi(\tau_1) \chi(\tau_s)\\
  & +2 \chi(\tau_2) \chi(\tau_3) + \ldots + 2 \chi(\tau_2) \chi(\tau_2) + \ldots + 2 \chi(\tau_{s-1}) \chi(\tau_s)
  \, .
\end{align*}
Therefore, the $s$-fold integral has two kinds of terms
\begin{align*}
  (a) &\quad \int_{-\infty}^{\infty} \chi^2(\tau_i) Q_s(\tau_1, \tau_2, \ldots, \tau_i, \ldots, \tau_s) d\tau_1 d\tau_2 \ldots d \tau_s\\
  (b) &\quad \int_{-\infty}^{\infty} \chi(\tau_i) \chi(\tau_j) Q_s(\tau_1,\tau_2, \ldots, \tau_s) d\tau_1 d\tau_2 \ldots d\tau_s
  \, .
\end{align*}
Using the symmetry properties of $Q_s$ one can transform these expressions. For $(a)$ we have
\begin{align*}
  & \int_{-\infty}^{\infty}\chi^2(\tau_i)Q_s(\tau_1,\tau_2, \ldots,\tau_i, \ldots, \tau_s)d\tau_1d\tau_2 \ldots d\tau_s \\
  \Leftrightarrow &\int_{-\infty}^{\infty} \chi^2(\tau_i) Q_s(\tau_i,\tau_2, \ldots, \tau_1, \ldots, \tau_s) \, d\tau_1 d\tau_2 \ldots d\tau_s \\
  \Leftrightarrow &\int_{-\infty}^{\infty} \chi^2(\tau_1) Q_s(\tau_1,\tau_2, \ldots, \tau_i, \ldots, \tau_s) \, d\tau_i d\tau_2 \ldots d\tau_1 \ldots d\tau_s \\
  \Leftrightarrow &\int_{t_a}^{t_b}\left[d\tau_1 \int_{-\infty}^{\infty}Q_s(\tau_1, \ldots, \tau_s) d\tau_2 \ldots d\tau_s \right]
  \, .
\end{align*}
In the same fashion we can transform the expression $(b)$ getting the following result
\begin{align*}
  & \int_{-\infty}^{\infty}\chi(\tau_i)\chi(\tau_j)Q_s(\tau_1,\tau_2,...,\tau_s)d\tau_1d\tau_2...d\tau_s \\
  \Leftrightarrow
  & \int_{t_a}^{t_b}\left[d\tau_1\int_{t_a}^{t_b}d\tau_2\int_{-\infty}^{\infty}Q_s(\tau_1,\tau_2,...,\tau_s)d\tau_3...d\tau_s\right]
  \, .
\end{align*}
In the end we can rewrite $\avgang{N^2}$
\begin{align*}
  \avgang{N^2}
  &= \sum_{s=1}^{\infty}\frac{1}{s!} \left( s\int_{t_a}^{t_b}\left[d\tau_1 \int_{-\infty}^{\infty}Q_s(\tau_1,...,\tau_s)d\tau_2...d\tau_s\right] + (s-1)s \int_{t_a}^{t_b} \left[d\tau_1\int_{t_a}^{t_b}d\tau_2\int_{-\infty}^{\infty}Q_s(\tau_1,\tau_2,...,\tau_s)d\tau_3...d\tau_s\right] \right) \\
  &= \sum_{s=1}^\infty \frac{1}{(s-1)!} \int_{t_a}^{t_b} \left[d\tau_1 \int_{-\infty}^\infty Q_s(\tau_1,...,\tau_s)d\tau_2...d\tau_s\right] + \sum_{s=2}^{\infty}\frac{1}{(s-2)!}\int_{t_a}^{t_b}\left[d\tau_1\int_{t_a}^{t_b}d\tau_2\int_{-\infty}^{\infty}Q_s(\tau_1,\tau_2,...,\tau_s)d\tau_3...d\tau_s\right] \\
  &= \langle N \rangle+\sum_{s=2}^{\infty}\frac{1}{(s-2)!}\int_{t_a}^{t_b}\left[d\tau_1\int_{t_a}^{t_b}d\tau_2\int_{-\infty}^{\infty}Q_s(\tau_1,\tau_2,...,\tau_s)d\tau_3...d\tau_s\right]
  \, .
\end{align*}
