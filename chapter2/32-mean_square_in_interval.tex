\leveldown{Mean square of N in given interval - pg. 32}

\leveldown{Problem}

Show that the mean square of N (number of dots) is given by the expression (1.7) in the book

\levelstay{Solution}
The number N of dots occurring in time follows the sequence
\begin{equation}
N \rightarrow \{0,\chi(\tau_1),\chi(\tau_1)+\chi(\tau_2),\chi(\tau_1)+\chi(\tau_1)+\chi(\tau_3),...\}
\end{equation}
Therefore
\begin{equation}
\langle N^2\rangle=\langle \left[\sum_{\sigma=1}^s \chi(\tau_{\sigma})\right]^2\rangle=\sum_{s=1}^{\infty}\frac{1}{s!}\int_{-\infty}^{\infty}\left[\sum_{\sigma=1}^s \chi(\tau_{\sigma})\right]^2Q_s(\tau_1,\tau_2,...,\tau_s)d\tau_1d\tau_2...d\tau_s
\end{equation}
The squared sum can be split up as
\begin{equation}
\begin{split}
&\left[ \sum_{\sigma=1}^s \chi(\tau_{\sigma}) \right]^2=\sum_{\sigma=1}^1 \chi^2(\tau_{\sigma})+2\chi(\tau_{1})\chi(\tau_{2})+...+2\chi(\tau_{1})\chi(\tau_{s})\\
&+2\chi(\tau_{2})\chi(\tau_{3})+...+2\chi(\tau_{2})\chi(\tau_{2})+...+2\chi(\tau_{s-1})\chi(\tau_{s})
\end{split}
\end{equation}
Thus the s-fold integral has two kinds of terms
\begin{equation}
\begin{split}
&(a)\quad \int_{-\infty}^{\infty}\chi^2(\tau_i)Q_s(\tau_1,\tau_2,...,\tau_i,...,\tau_s)d\tau_1d\tau_2...d\tau_s\\
&(b) \quad \int_{-\infty}^{\infty}\chi(\tau_i)\chi(\tau_j)Q_s(\tau_1,\tau_2,...,\tau_s)d\tau_1d\tau_2...d\tau_s 
\end{split}
\end{equation}
Using the symmetry properties of $Q_s$ one can transform these expressions. For $(a)$ we have
\begin{equation}
\begin{split}
& \int_{-\infty}^{\infty}\chi^2(\tau_i)Q_s(\tau_1,\tau_2,...,\tau_i,...,\tau_s)d\tau_1d\tau_2...d\tau_s\\
& \Leftrightarrow \int_{-\infty}^{\infty}\chi^2(\tau_i)Q_s(\tau_i,\tau_2,...,\tau_1,...,\tau_s)d\tau_1d\tau_2...d\tau_s \\
& \Leftrightarrow \int_{-\infty}^{\infty}\chi^2(\tau_1)Q_s(\tau_1,\tau_2,...,\tau_i,...,\tau_s)d\tau_id\tau_2...d\tau_1...d\tau_s \\
& \Leftrightarrow\int_{t_a}^{t_b}\left[d\tau_1 \int_{-\infty}^{\infty}Q_s(\tau_1,...,\tau_s)d\tau_2...d\tau_s \right]
\end{split}
\end{equation}
In the same fashion we can transform the expression $(b)$ getting the following result
\begin{equation}
\begin{split}
& \int_{-\infty}^{\infty}\chi(\tau_i)\chi(\tau_j)Q_s(\tau_1,\tau_2,...,\tau_s)d\tau_1d\tau_2...d\tau_s \\
& \Leftrightarrow \int_{t_a}^{t_b}\left[d\tau_1\int_{t_a}^{t_b}d\tau_2\int_{-\infty}^{\infty}Q_s(\tau_1,\tau_2,...,\tau_s)d\tau_3...d\tau_s\right] \\
\end{split}
\end{equation}
In the end we can rewrite $\langle N^2 \rangle$
\begin{equation}
\begin{split}
&\langle N^2 \rangle=\sum_{s=1}^{\infty}\frac{1}{s!}\{s\int_{t_a}^{t_b}\left[d\tau_1 \int_{-\infty}^{\infty}Q_s(\tau_1,...,\tau_s)d\tau_2...d\tau_s\right] \\
& +(s-1)s\int_{t_a}^{t_b}\left[d\tau_1\int_{t_a}^{t_b}d\tau_2\int_{-\infty}^{\infty}Q_s(\tau_1,\tau_2,...,\tau_s)d\tau_3...d\tau_s\right] \} \\
&\Leftrightarrow \langle N^2 \rangle=\sum_{s=1}^{\infty}\frac{1}{(s-1)!}\int_{t_a}^{t_b}\left[d\tau_1 \int_{-\infty}^{\infty}Q_s(\tau_1,...,\tau_s)d\tau_2...d\tau_s\right]\\
& +\sum_{s=2}^{\infty}\frac{1}{(s-2)!}\int_{t_a}^{t_b}\left[d\tau_1\int_{t_a}^{t_b}d\tau_2\int_{-\infty}^{\infty}Q_s(\tau_1,\tau_2,...,\tau_s)d\tau_3...d\tau_s\right] \\
& \Leftrightarrow \langle N^2 \rangle=\langle N \rangle+\sum_{s=2}^{\infty}\frac{1}{(s-2)!}\int_{t_a}^{t_b}\left[d\tau_1\int_{t_a}^{t_b}d\tau_2\int_{-\infty}^{\infty}Q_s(\tau_1,\tau_2,...,\tau_s)d\tau_3...d\tau_s\right] \\
\end{split}
\end{equation}