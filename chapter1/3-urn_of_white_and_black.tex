\leveldown{Urn of white and black - pg. 3}

\leveldown{Problem}

An urn contains a mixture of $N_w$ white balls and $N_b$ black ones.
I extract at random $M$ balls, without putting them back.
Show that the probability for having $n$ white balls among them is
\begin{equation*}
  P(n)
  = \binom{N_w}{n} \binom{N_b}{M-n} / \binom{N_w + N_b}{M}
  \, . \qquad \text{(``hypergeometric distribution'')}
\end{equation*}
This reduces to the volumes equation in the limit $N_w, N_b \rightarrow \infty$ with $N_w/N_b = \gamma$.

\levelstay{Solution}

We solve the problem by first assuming that the balls are all distinguishable, and that we draw them from the urn in a particular order.
We will then apply one correction for the fact that the black (white) balls are indistinguishable from each other and another to account for the ordering.

Suppose that the balls, in addition to having color, have some other trait that makes them each completely distinguishable, e.g. they could each be numbered.
Imagine a particular case where we extract $M$ balls one by one, getting first $n$ white balls and then $k$ black balls
\begin{equation*}
  w_1, w_2, \ldots , w_n , b_1, b_2, \ldots , b_k
  \, .
\end{equation*}
What is the probability of this precise string?
The probability of drawing ball $w_1$ on the first draw is $N_w / (N_w + N_b)$.
Then, with ball $w_1$ already removed, the probability of drawing ball $w_2$ on the second draw is $(N_w - 1)/(N_w + N_b - 1)$.
Continuing this reasoning, we find the probability $P$ of the string to be
\begin{align*}
  P =&
  \underbrace{
    \left(
      \frac{N_w}{N_w + N_b}
    \right)
    \left(
      \frac{N_w - 1}{N_w + N_b -1}
    \right)
    \cdots
    \left(
      \frac{N_w - (n - 1)}{N_w + N_b - n + 1}
    \right)
  }_\text{probability of white balls} \\
  & \underbrace{
    \left(
      \frac{N_b}{N_w + N_b - n}
    \right)
    \left(
      \frac{N_b - 1}{N_w + N_b - n - 1}
    \right)
    \cdots
    \left(
      \frac{N_b - (k - 1)}{N_w + N_b - n - k + 1}
    \right)
  }_\text{probability of black balls} \\
  =&
  \frac{N_w!}{(N_w - n)!}
  \frac{N_b!}{(N_b - k)!}
  \frac{(N_w + N_b -n - k)!}{(N_w + N_b)!} \, .
\end{align*}
Now of course, we're really trying to find the probability of getting $n$ indistinguishable white balls and $k$ indistinguishable black ones.
Therefore, shuffling the $n$ white balls or $k$ black balls among themselves does not result in a uniquely distinguishable arrangement of the draws.
To correct for the overcounting, we should divide by $n! k!$.
Furthermore, we don't care what order we draw the balls in, so we should multiply by $(n + k)!$.
The result is
\begin{equation*}
  \frac{N_w!}{(N_w - n)! n!} \frac{N_b!}{(N_b - k)k!} \frac{(n + k)!(N_w + N_b - (n + k))!}{(N_w + N_b)!} \, .
\end{equation*}
Using the binomial notation we can write
\begin{equation*}
  \binom{N_w}{n} \binom{N_b}{k} / \binom{N_w + N_b}{k + n} 
\end{equation*}
which is equivalent to what we are trying to show because $M = n + k$.
