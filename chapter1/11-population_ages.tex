\leveldown{Conditional life time probabilities - pg. 11}

\leveldown{Problem}

The probability distribution of lifetimes in a population is $P(t)$.
Show that the conditional probability for individuals of age $\tau$ is
\begin{equation*}
  P_{\text{die at age}|\text{alive at age}}(t | \tau) = P(t) / \int_\tau^\infty P(t') \, dt' \quad (t > \tau)
  \, .
\end{equation*}
Note that in the case $P(t) = \gamma e^{-\gamma t}$ one has $P_{\text{die at age}|\text{alive at age}}(t|\tau) = P(t - \tau)$: the survival chance is independent of age.
Show that this is the only $P$ for which that is true.

\levelstay{Solution}

Use Bayes's rule,
\begin{equation*}
  P_{\text{die at age}|\text{alive at age}}(t | \tau)
  = \frac{P_{\text{alive at age}|\text{die at age}}(\tau, t)P(t)}{P_{\text{alive at age}}(\tau)}
  \, .
\end{equation*}
If your age is $\tau$ that means you die at some time $t$ such that $t > \tau$.
Therefore, the conditional probabilities are nonzero only when $t > \tau$ and so we work in that case for the rest of the problem.

For the denominator, the probability that a given person is still alive at age $\tau$ is the sum of probabilities that they die at some time $t$ after $\tau$, i.e.
\begin{equation*}
  P_\text{alive at age }(\tau) = \int_\tau^\infty P(t') \, dt'
  \, .
\end{equation*}
For the numerator, in the case $t > \tau$, the conditional probability is identically equal to one
\begin{displaymath}
  P_{\text{alive at age}|\text{die at age}}(\tau, t) = 1
\end{displaymath}
so
\begin{equation*}
  P_{\text{die at age}|\text{alive at age}}(t | \tau)
  = \frac{P(t)}{\int_\tau^\infty P(t') \, dt'}
\end{equation*}
as we wanted to show.

Now we suppose $P(t) = \gamma e^{-\gamma t}$.
Then the integral is
\begin{align*}
  \int_\tau^\infty P(t') \, dt'
  &= \gamma \left. \frac{e^{-\gamma t}}{-\gamma} \right|_\tau^\infty \\
  &= e^{-\gamma \tau}
  \, .
\end{align*}
Therefore, the conditional probability is
\begin{equation*}
  P_{\text{die at age}|\text{alive at age}}(t | \tau)
  = \frac{\gamma e^{-\gamma t}}{e^{-\gamma \tau}} = \gamma e^{-\gamma (t - \tau)}
  = P( t - \tau)
\end{equation*}
which is what we wanted to show.

To show that there is only one possible $P(t)$ with the property we just demonstrated, we use calculus:
\begin{align*}
  P(t - \tau) &= \frac{P(t)}{\int_\tau^\infty P(t') \, dt'} \\
  \int_\tau^\infty P(t') \, dt' &= \frac{P(t)}{P(t - \tau)} \\
  (\text{differentiate w.r.t. }\tau) \qquad -P(\tau) &= -\frac{P(t)}{P(t - \tau)^2}
  \left( - P'(t-\tau) \right) \\
  P'(t - \tau) &= - \frac{P(\tau)P(t - \tau)^2}{P(t)} \\
  (\text{set }\tau=0) \quad P'(t) &= - P(0)P(t) \\
  P(t) &= P(0) e^{-P(0) \, t} \, .
\end{align*}
Denoting $P(0)$ as $\gamma$ gives the result.
